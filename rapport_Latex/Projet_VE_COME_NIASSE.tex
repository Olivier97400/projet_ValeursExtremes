\documentclass[a4paper,french,10pt]{article}
\usepackage{homework}
\usepackage{diagbox}

% change le nom de la table des matières
\addto\captionsfrench{\renewcommand*\contentsname{Sommaire}}

\lstdefinelanguage{R}%
{morekeywords={function,for,in,if,elseif,else,TRUE,FALSE,%
		return, while, diag, sum, sqrt, nrow, ncol, par, plot, cbind, rep, as, survdiff, survreg, ifelse, anova,
		row, names, colnames, mean, data, frame, model, in, list, rexp, rpois, summary,
		matrix, TRUE, FALSE, for, if, else, function, NA, print, survfit, Surv, rho, ggplot},%
	sensitive=true,%
	morecomment=[l]{\#},%
	morestring=[s]{"}{"},%
	morestring=[s]{'}{'},%
}[keywords,comments,strings]%

\lstset{%
	language         = R,
	basicstyle       = \ttfamily,
	keywordstyle     = \bfseries\color{blue},
	stringstyle      = \color{magenta},
	commentstyle     = \color{olive},
	showstringspaces = false,
}

\begin{document}
	
	% Blank out the traditional title page
	\title{\vspace{-1in}} % no title name
	\author{} % no author name
	\date{} % no date listed
	\maketitle % makes this a title page
	
	% Use custom title macro instead
	\usebox{\myReportTitle}
	\vspace{1in} % spacing below title header
	
	% Assignment title
	{\centering \huge \assignmentName \par}
	{\centering \noindent\rule{4in}{0.1pt} \par}
	\vspace{0.05in}
	{\centering \courseCode~: \courseName~ \par}
	{\centering Rédigé le \pubDate\ en \LaTeX \par}
	\vspace{1in}
	
	% Table of Contents
	\tableofcontents
	\newpage
	
	%----------------------------------------------------------------------------------------
	%	EXERCICE 1
	%----------------------------------------------------------------------------------------
	
\section{Introduction}
Gueladio wolof	

%\lstinputlisting[language=R, firstline=2, lastline=13]{code/TP3_NIASSE_COME.R}

%\begin{figure}[htp] 
%	\centering
%	\includegraphics[scale=0.45]{images/courbe1.png}
%	\caption{Comparaison entre l'estimation de la fonction de Nelson-Aelen et celle de la loi de Weibull avec un taux de censure de 13,8\%}
%\end{figure}

%\begin{figure}[htp] 
%	\centering
%	\subfloat[Taux de censure à 25\%]{%
%			\includegraphics[scale=0.4]{images/c25.png}%
%		}%
%	\hfill%
%	\subfloat[Taux de censure à 34\%]{%
%			\includegraphics[scale=0.4]{images/c30.png}%
%		}%
%	\caption{Estimation de la fonction de Nelson-Aelen et de la loi de Weibull}
%\end{figure}

\end{document}
